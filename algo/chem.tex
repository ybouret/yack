\documentclass[aps,12pt]{revtex4}
\usepackage[a4paper]{geometry}
\usepackage{graphicx}
\usepackage{amssymb,amsfonts,amsmath,amsthm}
\usepackage{bm}
\usepackage{pslatex}
\usepackage{mathptmx}
\usepackage{bookman}
\usepackage{chemarr}

 	 
\begin{document}

\section{Notations}

Let $A_1,\ldots,A_M$ be $M$ species involved in $N$ equilibria:
\begin{equation}
	\forall i\in[1:N], \; \sum_j \nu_{i,j} A_j = 0
\end{equation}
and
\begin{equation}
	\forall i\in[1:N], \; 
	\Gamma_i = K_i \prod_{j=1}^{M} [A_j]^{\nu^r_{ij}} 
	- \prod_{j=1}^{M} [A_j]^{\nu^p_{ij}} = 0
\end{equation}

\begin{itemize}
\item $\Gamma_i$ has the sign of $\Delta_r G_i$.
\item The unit of $\Gamma_i$ is $C_0^{\sum_j \nu^p_{ij} }$
\item The unit of $K_i$ is $C_0^{\Delta_r \nu}$
\end{itemize}


\section{Derivatives}
We express the derivative of a product w.r.t. one term:
\begin{equation}
	\partial_{A_k} \left( \prod_{j=1}^{M} [A_j]^{\nu_{j}} \right)  =
	\nu_k [A_k]^{\nu_k-1} \left( \prod_{j=1,j\not=k}^{M} [A_j]^{\nu_{j}} \right)
\end{equation}

Then:
\begin{equation}
	\partial^2_{A_k, A_k} \left( \prod_{j=1}^{M} [A_j]^{\nu_{j}} \right)  =
	\nu_k (\nu_k-1) [A_k]^{\nu_k-2} \left( \prod_{j=1,j\not=k}^{M} [A_j]^{\nu_{j}} \right)
\end{equation}
and:
\begin{equation}
	\partial^2_{A_k\not=A_l} \left( \prod_{j=1}^{M} [A_j]^{\nu_{j}} \right)  =
	\nu_k [A_k]^{\nu_k-1} \nu_l [A_l]^{\nu_l-1}\left( \prod_{j=1,j\not=k,j\not=l}^{M} [A_j]^{\nu_{j}} \right)
\end{equation}

\begin{equation}
	\vec{\Psi}_i = \partial_{\vec{C}} \Gamma_i
\end{equation}

%\section{Subtracting Equation}
%
%Let $\vec{V}_1\not=\vec{0}$ and $\vec{V}_2\not=\vec{0}$.
%The smallest weighted difference is:
%\begin{equation}
%	\delta \vec{V} = \dfrac{1}{\left(\vec{V}_1+\vec{V}_2\right)^2} \left[ 
%	\langle \vec{V}_2 | \vec{V}_1+\vec{V}_2 \rangle \vec{V}_1 - \langle \vec{V}_1 | \vec{V}_1+\vec{V}_2 \rangle \vec{V}_2
%	\right]
%\end{equation}

\section{1D-Solving}

\subsection{Equation}
Let us take $\Gamma_i$ (with a least a product or a reactant).
Then there exist a unique $\Xi_i$ for which:
\begin{equation}
\left\lbrace
\begin{array}{rcl}
	\Gamma_i(\vec{C} + \Xi_i \vec{\nu}_i ) & = & 0\\
	 \Xi_i \times \Gamma_i(\vec{C}) &\geq  &0\\
\end{array}
\right.
\end{equation}
 
Indeed, $\Xi_i$ is solution of:
\begin{equation}
	 F_i(\vec{C},X) =  K_i \prod_{j=1}^{M} \left([A_j] - X \nu^r_{ij}\right)^{\nu^r_{ij}} 
	- \prod_{j=1}^{M} \left([A_j] + X \nu^p_{ij} \right)^{\nu^p_{ij}} = 0
\end{equation} 

And each $\Xi_i$ respects all the limiting extents.

\subsection{Where to look for}
Let us set:
\begin{equation}
\left\lbrace
\begin{array}{rcll}
	P_i(\vec{C},X) & = & \displaystyle \prod_{j=1}^{M} \left([A_j] + X \nu^p_{ij} \right)^{\nu^p_{ij}}, & \partial_X P_i(\vec{C},X) \geq 0 \\
	\\
	R_i(\vec{C},X) & = & \displaystyle K_i \prod_{j=1}^{M} \left([A_j] - X \nu^r_{ij}\right)^{\nu^r_{ij}}, & \partial_X R_i(\vec{C},X) \leq 0 \\
	\\
	 F_i(\vec{C},X) & = & R_i(\vec{C},X)  - P_i(\vec{C},X), & \partial_X F_i(\vec{C},X) < 0\\
\end{array}
\right.
\end{equation}

Let
\begin{equation}
	s_i = \mathrm{sign}\lbrack F_i(\vec{C},0) \rbrack
\end{equation}

\begin{itemize}
\item If $s_i=0$ then $X=0$ is solution at $\vec{C}$...

\item If $s_i>0$, then we look for $X>0$

\item If $s_i<0$, then we look for $X<0$

\end{itemize}

\subsection{Looking for $X\not=0$}
	
\subsubsection{Limited by BOTH}

\begin{itemize}
\item Looking for $X<0$:
$$
X \in \rbrack -\xi_p; 0 \lbrack
$$
\item Looking for $X>0$:
$$
X \in \rbrack 0; \xi_r \lbrack
$$
\end{itemize}

\subsubsection{Limited by REACTANTS only}
This is a "combination only" reaction:
\begin{itemize}
\item
$$
	\Delta_r \nu = - \Delta_r \nu^r < 0 
$$
\item $$P_i \equiv 1$$
\item $$X \in \rbrack -\infty; \xi_r \lbrack$$
\item $$F_i(\vec{C},\xi_r)=-1$$
\item Looking for $X>0$:
 $$
 \left\lbrace
 \begin{array}{rcl}
 	X &\in& \rbrack 0;\xi_r \lbrack\\
	F &\in& \rbrack F_i(\vec{C},0)>0 ; -1 \lbrack\\
\end{array}
\right.
 $$
 \item Looking for $X<0$:\\
 We use $S_i=K_i^{\frac{1}{\Delta_r \nu}}=K_i^{\frac{-1}{\Delta_r \nu^r}}$
 and look for $n$ such that:
 $$
 	\left\lbrace
 \begin{array}{rcl}
 	X &\in& \rbrack -2^n S_i;0 \lbrack\\
	F &\in& \rbrack F_i(\vec{C},-2^nS_i)>0 ; F_i(\vec{C},0)<0 \lbrack\\
\end{array}
\right.
 $$
\end{itemize}
 	
\subsubsection{Limited by PRODUCTS only}
This is a "dissociation only" reaction:
\begin{itemize}
\item
$$
	\Delta_r \nu =  \Delta_r \nu^p > 0 
$$
\item
$$
	R_i \equiv K_i
$$
\item
$$
	X \in \rbrack -\xi_p; +\infty \lbrack
$$
\item Looking for $X<0$:
$$
 \left\lbrace
 \begin{array}{rcl}
 	X &\in& \rbrack -\xi_p; 0 \lbrack\\
	F &\in& \rbrack K_i; F_i(\vec{C},0)<0 \lbrack\\
\end{array}
\right.
$$
\item Looking for $X>0$:\\
 We use $S_i=K_i^{\frac{1}{\Delta_r \nu}}=K_i^{\frac{1}{\Delta_r \nu^p}}$
 and look for $n$ such that:
 $$
 	\left\lbrace
 \begin{array}{rcl}
 	X &\in& \rbrack 0; 2^n S_i\lbrack\\
	F &\in& \rbrack F_i(\vec{C},0)>0; F_i(\vec{C},2^nS_i)<0 \lbrack\\
\end{array}
\right.
 $$



\end{itemize}
	
	
\section{Multiple Solving}

\subsection{Extension in 1D}

\begin{equation}
	\Gamma_i\left(\vec{C}+\xi_i \vec{\nu}_i  + \sum_{j\not= i} \xi_j \vec{\nu}_j \right) 	\simeq 
	\underbrace{\langle \vec{\nu}_i \vert \vec{\Psi}_i' \rangle}_{\sigma_i<0} (\xi_i-\Xi_i) + \square   (\xi_i-\Xi_i)^2
	+ \sum_{j\not=i}  \langle \vec{\Psi}_j \vert \vec{\nu}_j \rangle \xi_j
\end{equation}

\begin{equation}
	\dfrac{\Gamma_i\left(\vec{C}+\xi_i \vec{\nu}_i  + \sum_{j\not= i} \xi_j \vec{\nu}_j \right) }{\sigma_i}  \simeq
	\left(\xi_i - \Xi_i\right) + \dfrac{\square}{\sigma_i}\left(\xi_i - \Xi_i\right)^2 + \sum_{j\not=i}  \dfrac{\langle \vec{\Psi}_j \vert \vec{\nu}_j \rangle}{\sigma_i} \xi_j
\end{equation}


\begin{equation}
	-\dfrac{\Gamma_i\left(\vec{C}+\xi_i \vec{\nu}_i  + \sum_{j\not= i} \xi_j \vec{\nu}_j \right) }{\sigma_i}  \simeq
	 \Xi_i + A_i \left(\xi_i - \Xi_i\right)^2 - \left[ \xi_i + \underbrace{\sum_{j\not=i}  \dfrac{\langle \vec{\Psi}_j \vert \vec{\nu}_j \rangle}{\sigma_i} \xi_j}_{\lambda_i} \right]
\end{equation}

\begin{equation}
		-\dfrac{\Gamma_i}{\sigma_i} = \Xi_i + A_i \Xi_i^2 = \gamma_i \text{ (is a concentration)}
\end{equation}

\begin{equation}
	(1-\alpha_i) \gamma_i = \Xi_i + A_i \left(\xi_i - \Xi_i\right)^2 - (\xi_i+\lambda_i)
\end{equation}

\begin{equation}
	(1-\alpha_i) (\Xi_i+A_i\Xi_i^2) = \Xi_i + A_i \left(\xi_i - \Xi_i\right)^2 - (\xi_i+\lambda_i)
\end{equation}

\begin{equation}
	0  =  A_i \left[ \left(\xi_i - \Xi_i\right)^2 - (1-\alpha_i) \Xi_i^2\right] - \underbrace{(\xi_i+\lambda_i)}_{\langle \vec{\Omega}_i\vert \vec{\xi} \rangle} + \alpha_i \Xi_i 
\end{equation}

\begin{itemize}
\item	$\vec{\alpha}=\vec{0}$
	\begin{equation}
	0 = A_i \xi_i (\xi_i-2\Xi_i) - \langle \vec{\Omega}_i\vert \vec{\xi} \rangle
	\end{equation}
	\begin{equation}
	\left[\bm{\Omega} + \bm{\Delta}(\vec{\xi},\vec{\Xi})\right] \vec{\xi} = 0
	\end{equation}
	
\item	$\vec{\alpha}=\vec{1}$
	\begin{equation}
	0=\Xi_i + A_i(\xi_i-\Xi_i)^2 - \langle \vec{\Omega}_i\vert \vec{\xi} \rangle
	\end{equation}
\end{itemize}

\end{document}


\begin{equation}
	\gamma_i - \alpha_i \gamma_i = \Xi_i + A_i (\xi_i^2 - 2 \xi_i \Xi_i + \Xi_i^2) - (\xi_i+\lambda_i)
\end{equation}


\begin{equation}
	 	- \alpha_i \gamma_i =  A_i (\xi_i^2 - 2 \xi_i \Xi_i ) - (\xi_i+\lambda_i)
\end{equation}

\begin{equation}
	A_i \xi_i^2 - \xi_i (2\Xi_iA_i+1) - \lambda_i + \alpha_i \underbrace{\Xi_i(1 + A_i \Xi_i)}_{\gamma_i} = 0
\end{equation}

\begin{equation}
\begin{array}{rcl}
	\Delta_i & = & (1+2\Xi_iA_i)^2 - 4 \alpha_i A_i \Xi_i (1+A_i\Xi_i) + 4 A_i \lambda_i\\
	& = & (1-\alpha_i) (1+2\Xi_iA_i)^2 + \alpha_i +  4 A_i \lambda_i\\
	& = & 1 + 4 A_i \left[ \lambda_i + (1-\alpha_i) \underbrace{(\Xi_i+A_i\Xi_i^2)}_{\gamma_i}\right] \\
\end{array}
\end{equation}


\begin{equation}
	A_i (\xi_i^2 - 2 \Xi _i \xi_i + \alpha_i \Xi_i^2 )    - \langle \vec{\Omega}_i\vert \vec{\xi} \rangle + \alpha_i  \Xi_i  = 0
\end{equation}

\end{document}

 \begin{equation}
 \begin{array}{rcl}
	\Gamma_i & = & A_i \Xi_i^2  -\sigma_i \Xi_i\\
	\Gamma_i \Xi_i \geq 0 & \Rightarrow &  P_i = A_i \Xi_i - \sigma_i \geq 0\\
\end{array}
 \end{equation}
 
\begin{equation}
	(1-\alpha_i)\Gamma_i = \sigma_i (\xi_i-\Xi_i) + A_i (\xi_i-\Xi_i)^2 
	+ \underbrace{\sum_{j\not=i} \underbrace{\langle \vec{\Psi}_i \vert \vec{\nu}_j \rangle}_{\sigma_{ij}} \xi_j}_{\lambda_i}
\end{equation}
 
\begin{equation}
	\Gamma_i - \alpha_i \Gamma_i = \sigma_i \xi_i - \sigma_i \Xi_i + A_i \xi_i^2  - 2 A_i \xi_i \Xi_i + A_i \Xi_i^2 + \lambda_i
\end{equation}
  
 \begin{equation}
 - \alpha_i \Gamma_i = -\alpha_i\left( A_i \Xi_i^2  -\sigma_i \Xi_i\right) = \sigma_i \xi_i   + A_i \xi_i^2  - 2 A_i \xi_i \Xi_i   + \lambda_i
\end{equation}

\begin{equation}
	0 = A_i \xi_i^2  - \underbrace{[A_i\Xi_i + (A_i\Xi_i - \sigma_i)]}_{B_i=A_i\Xi_i + P_i} \xi_i + \underbrace{\lambda_i  + \alpha_i \Xi_i \underbrace{(A_i \Xi_i  -\sigma_i)}_{P_i}}_{C_i}
\end{equation}

\begin{quote}
\it
The goal is to find a linear relation between $\xi_i$ and $\lambda_i$.
\end{quote}

\begin{equation}
\begin{array}{rcl}
	\Delta_i & = &[A_i\Xi_i + (A_i\Xi_i - \sigma_i)]^2 - 4 \alpha_i A_i \Xi_i \,(A_i \Xi_i  -\sigma_i)
	- 4 A_i \lambda_i\\
	& = & (1-\alpha_i) [A_i\Xi_i + (A_i\Xi_i - \sigma_i)]^2 + \alpha_i \sigma_i^2 - 4 A_i \lambda_i\\
	& = & \sigma_i^2 + 4 A_i \left[ (1-\alpha_i) \Xi_i \left( A_i\Xi_i - \sigma_i  \right) -  \lambda_i \right]\\
	& = & \sigma_i^2 + 4 A_i \left[  \Xi_i P_i - C_i \right]\\
	& = & \sigma_i^2\left( 1 + \dfrac{4A_i}{\sigma_i^2}\left[  \Xi_i P_i - C_i \right] \right)\\
\end{array}
\end{equation}

\begin{equation}
	\xi_i^\pm  = \dfrac{2}{B_i \pm \sqrt\Delta_i }   C_i  
\end{equation}

\begin{equation}
	0 = \xi_i^\pm + \dfrac{2}{\sqrt\Delta_i \pm B_i} C_i 
\end{equation}

 
\end{document}
