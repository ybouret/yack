\documentclass[aps,12pt]{revtex4}
\usepackage[a4paper]{geometry}
\usepackage{graphicx}
\usepackage{amssymb,amsfonts,amsmath,amsthm}
\usepackage{bm}
\usepackage{pslatex}
\usepackage{mathptmx}
\usepackage{bookman}
\usepackage{chemarr}
\usepackage{tikz}

\newcommand*\circled[1]{\tikz[baseline=(char.base)]{
            \node[shape=circle,draw,inner sep=2pt] (char) {#1};}}
 	 
\begin{document}

\section{Notations}

Let $A_1,\ldots,A_M$ be $M$ species involved in $N$ equilibria:
\begin{equation}
	\forall i\in[1:N], \; \sum_j \nu_{i,j} A_j = 0
\end{equation}
and
\begin{equation}
	\forall i\in[1:N], \; 
	\Gamma_i = K_i \prod_{j=1}^{M} [A_j]^{\nu^r_{ij}} 
	- \prod_{j=1}^{M} [A_j]^{\nu^p_{ij}} = 0
\end{equation}

\begin{itemize}
\item $\Gamma_i$ has the sign of $\Delta_r G_i$.
\item The unit of $\Gamma_i$ is $C_0^{\sum_j \nu^p_{ij} }$
\item The unit of $K_i$ is $C_0^{\Delta_r \nu}$
\end{itemize}


\section{Derivatives}
We express the derivative of a product w.r.t. one term:
\begin{equation}
	\partial_{A_k} \left( \prod_{j=1}^{M} [A_j]^{\nu_{j}} \right)  =
	\nu_k [A_k]^{\nu_k-1} \left( \prod_{j=1,j\not=k}^{M} [A_j]^{\nu_{j}} \right)
\end{equation}

Then:
\begin{equation}
	\partial^2_{A_k, A_k} \left( \prod_{j=1}^{M} [A_j]^{\nu_{j}} \right)  =
	\nu_k (\nu_k-1) [A_k]^{\nu_k-2} \left( \prod_{j=1,j\not=k}^{M} [A_j]^{\nu_{j}} \right)
\end{equation}
and:
\begin{equation}
	\partial^2_{A_k\not=A_l} \left( \prod_{j=1}^{M} [A_j]^{\nu_{j}} \right)  =
	\nu_k [A_k]^{\nu_k-1} \nu_l [A_l]^{\nu_l-1}\left( \prod_{j=1,j\not=k,j\not=l}^{M} [A_j]^{\nu_{j}} \right)
\end{equation}

\begin{equation}
	\vec{\Psi}_i = \partial_{\vec{C}} \Gamma_i
\end{equation}

%\section{Subtracting Equation}
%
%Let $\vec{V}_1\not=\vec{0}$ and $\vec{V}_2\not=\vec{0}$.
%The smallest weighted difference is:
%\begin{equation}
%	\delta \vec{V} = \dfrac{1}{\left(\vec{V}_1+\vec{V}_2\right)^2} \left[ 
%	\langle \vec{V}_2 | \vec{V}_1+\vec{V}_2 \rangle \vec{V}_1 - \langle \vec{V}_1 | \vec{V}_1+\vec{V}_2 \rangle \vec{V}_2
%	\right]
%\end{equation}

\section{1D-Solving}

\subsection{Equation}
Let us take $\Gamma_i$ (with a least a product or a reactant).
Then there exist a unique $\Xi_i$ for which:
\begin{equation}
\left\lbrace
\begin{array}{rcl}
	\Gamma_i(\vec{C} + \Xi_i \vec{\nu}_i ) & = & 0\\
	 \Xi_i \times \Gamma_i(\vec{C}) &\geq  &0\\
\end{array}
\right.
\end{equation}
 
Indeed, $\Xi_i$ is solution of:
\begin{equation}
	 F_i(\vec{C},X) =  K_i \prod_{j=1}^{M} \left([A_j] - X \nu^r_{ij}\right)^{\nu^r_{ij}} 
	- \prod_{j=1}^{M} \left([A_j] + X \nu^p_{ij} \right)^{\nu^p_{ij}} = 0
\end{equation} 

And each $\Xi_i$ respects all the limiting extents.

\subsection{Where to look for}
Let us set:
\begin{equation}
\left\lbrace
\begin{array}{rcll}
	P_i(\vec{C},X) & = & \displaystyle \prod_{j=1}^{M} \left([A_j] + X \nu^p_{ij} \right)^{\nu^p_{ij}}, & \partial_X P_i(\vec{C},X) \geq 0 \\
	\\
	R_i(\vec{C},X) & = & \displaystyle K_i \prod_{j=1}^{M} \left([A_j] - X \nu^r_{ij}\right)^{\nu^r_{ij}}, & \partial_X R_i(\vec{C},X) \leq 0 \\
	\\
	 F_i(\vec{C},X) & = & R_i(\vec{C},X)  - P_i(\vec{C},X), & \partial_X F_i(\vec{C},X) < 0\\
\end{array}
\right.
\end{equation}

Let
\begin{equation}
	s_i = \mathrm{sign}\lbrack F_i(\vec{C},0) \rbrack
\end{equation}

\begin{itemize}
\item If $s_i=0$ then $X=0$ is solution at $\vec{C}$...

\item If $s_i>0$, then we look for $X>0$

\item If $s_i<0$, then we look for $X<0$

\end{itemize}

\subsection{Looking for $X\not=0$}
	
\subsubsection{Limited by BOTH}

\begin{itemize}
\item Looking for $X<0$:
$$
X \in \rbrack -\xi_p; 0 \lbrack
$$
\item Looking for $X>0$:
$$
X \in \rbrack 0; \xi_r \lbrack
$$
\end{itemize}

\subsubsection{Limited by REACTANTS only}
This is a "combination only" reaction:
\begin{itemize}
\item
$$
	\Delta_r \nu = - \Delta_r \nu^r < 0 
$$
\item $$P_i \equiv 1$$
\item $$X \in \rbrack -\infty; \xi_r \lbrack$$
\item $$F_i(\vec{C},\xi_r)=-1$$
\item Looking for $X>0$:
 $$
 \left\lbrace
 \begin{array}{rcl}
 	X &\in& \rbrack 0;\xi_r \lbrack\\
	F &\in& \rbrack F_i(\vec{C},0)>0 ; -1 \lbrack\\
\end{array}
\right.
 $$
 \item Looking for $X<0$:\\
 We use $S_i=K_i^{\frac{1}{\Delta_r \nu}}=K_i^{\frac{-1}{\Delta_r \nu^r}}$
 and look for $n$ such that:
 $$
 	\left\lbrace
 \begin{array}{rcl}
 	X &\in& \rbrack -2^n S_i;0 \lbrack\\
	F &\in& \rbrack F_i(\vec{C},-2^nS_i)>0 ; F_i(\vec{C},0)<0 \lbrack\\
\end{array}
\right.
 $$
\end{itemize}
 	
\subsubsection{Limited by PRODUCTS only}
This is a "dissociation only" reaction:
\begin{itemize}
\item
$$
	\Delta_r \nu =  \Delta_r \nu^p > 0 
$$
\item
$$
	R_i \equiv K_i
$$
\item
$$
	X \in \rbrack -\xi_p; +\infty \lbrack
$$
\item Looking for $X<0$:
$$
 \left\lbrace
 \begin{array}{rcl}
 	X &\in& \rbrack -\xi_p; 0 \lbrack\\
	F &\in& \rbrack K_i; F_i(\vec{C},0)<0 \lbrack\\
\end{array}
\right.
$$
\item Looking for $X>0$:\\
 We use $S_i=K_i^{\frac{1}{\Delta_r \nu}}=K_i^{\frac{1}{\Delta_r \nu^p}}$
 and look for $n$ such that:
 $$
 	\left\lbrace
 \begin{array}{rcl}
 	X &\in& \rbrack 0; 2^n S_i\lbrack\\
	F &\in& \rbrack F_i(\vec{C},0)>0; F_i(\vec{C},2^nS_i)<0 \lbrack\\
\end{array}
\right.
 $$



\end{itemize}
	
	
\section{Multiple Solving}

\subsection{Extension in 1D}

We know $\Gamma_i$ is a {\bf decreasing} polynomial with degree:
\begin{equation}
	d_i = \max\left(\sum_j \nu_{ij}^p,\sum_j \nu_{ij}^pr\right)
\end{equation}
and $\Xi_i$ is a first order zero of this polynomial. 

The partial derivative is:
\begin{equation}
	\partial_{\xi_i} \Gamma_i =  \langle \vec{\Psi}_{\xi_i} \vert \vec{\nu}_i \rangle
\end{equation}

Around $\Xi_i$:
\begin{equation}
\Gamma_i\left(\vec{C}+\xi_i \vec{\nu}_i \right) \simeq \underbrace{\langle \vec{\nu}_i \vert \vec{\Psi}_i' \rangle}_{\sigma'_i<0} (\xi_i-\Xi_i)
\end{equation}

Around $0$:
\begin{equation}
\Gamma_i\left(\vec{C}+\xi_i \vec{\nu}_i \right) \simeq \underbrace{\langle \vec{\nu}_i \vert \vec{\Psi}_i \rangle}_{\sigma_i \leq 0}\xi_i + \Gamma_i
\end{equation}



\begin{equation}
	\Gamma_i\left(\vec{C}+\xi_i \vec{\nu}_i \right) = \sigma_i'\left(\xi_i - \Xi_i\right) \left[ -\dfrac{\Gamma_i}{\Xi_i\sigma_i'} + \ldots \right]
\end{equation}

\begin{equation}
	\Gamma_i \left( \vec{C}+\xi_i \vec{\nu}_i + \sum_{j\not=i} \nu_j \vec{\xi}_j \right) = 
\end{equation}


\end{document}



 