\documentclass[aps,12pt]{revtex4}
\usepackage[a4paper]{geometry}
\usepackage{graphicx}
\usepackage{amssymb,amsfonts,amsmath,amsthm}
\usepackage{bm}
\usepackage{pslatex}
\usepackage{mathptmx}
\usepackage{bookman}

 	 
\begin{document}

Let $A_1,\ldots,A_M$ be $M$ species involved in $N$ equilibria:
\begin{equation}
	\forall i\in[1:N], \; \sum_j \nu_{i,j} A_j = 0
\end{equation}
and
\begin{equation}
	\forall i\in[1:N], \; 
	\Gamma_i = K_i \prod_{j=1}^{M} [A_j]^{\nu^r_{i,j}} 
	- \prod_{j=1}^{M} [A_j]^{\nu^p_{i,j}} = 0
\end{equation}

We express the derivative of a product w.r.t. one term:
\begin{equation}
	\partial_{A_k} \left( \prod_{j=1}^{M} [A_j]^{\nu_{j}} \right)  =
	\nu_k [A_k]^{\nu_k-1} \left( \prod_{j=1,j\not=k}^{M} [A_j]^{\nu_{j}} \right)
\end{equation}

For a given set of concentrations, $\Gamma_i$ is proportional
to the rate of the reaction corresponding to the equilibrium:
for a given amount of time:
\begin{equation}
	\delta \vec{C} = \bm{\nu}^T
	\begin{bmatrix}
	\lambda_1 & & \\
	 & \ddots & \\
	 & & \lambda_N \\
	\end{bmatrix}
	\vec{\Gamma}
	= \bm{\Omega} \vec{\lambda}, \;\; \bm{\Omega}_{ij} = \left(\bm{\nu}^T\right)_{ij} \Gamma_j
\end{equation}



\end{document}