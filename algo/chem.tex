\documentclass[aps,12pt]{revtex4}
\usepackage[a4paper]{geometry}
\usepackage{graphicx}
\usepackage{amssymb,amsfonts,amsmath,amsthm}
\usepackage{bm}
\usepackage{pslatex}
\usepackage{mathptmx}
\usepackage{bookman}
\usepackage{chemarr}

 	 
\begin{document}

\section{Notations}

Let $A_1,\ldots,A_M$ be $M$ species involved in $N$ equilibria:
\begin{equation}
	\forall i\in[1:N], \; \sum_j \nu_{i,j} A_j = 0
\end{equation}
and
\begin{equation}
	\forall i\in[1:N], \; 
	\Gamma_i = K_i \prod_{j=1}^{M} [A_j]^{\nu^r_{ij}} 
	- \prod_{j=1}^{M} [A_j]^{\nu^p_{ij}} = 0
\end{equation}

\begin{itemize}
\item $\Gamma_i$ has the sign of $\Delta_r G_i$.
\item The unit of $\Gamma_i$ is $C_0^{\sum_j \nu^p_{ij} }$
\item The unit of $K_i$ is $C_0^{\Delta_r \nu}$
\end{itemize}


\section{Derivatives}
We express the derivative of a product w.r.t. one term:
\begin{equation}
	\partial_{A_k} \left( \prod_{j=1}^{M} [A_j]^{\nu_{j}} \right)  =
	\nu_k [A_k]^{\nu_k-1} \left( \prod_{j=1,j\not=k}^{M} [A_j]^{\nu_{j}} \right)
\end{equation}

Then:
\begin{equation}
	\partial^2_{A_k, A_k} \left( \prod_{j=1}^{M} [A_j]^{\nu_{j}} \right)  =
	\nu_k (\nu_k-1) [A_k]^{\nu_k-2} \left( \prod_{j=1,j\not=k}^{M} [A_j]^{\nu_{j}} \right)
\end{equation}
and:
\begin{equation}
	\partial^2_{A_k\not=A_l} \left( \prod_{j=1}^{M} [A_j]^{\nu_{j}} \right)  =
	\nu_k [A_k]^{\nu_k-1} \nu_l [A_l]^{\nu_l-1}\left( \prod_{j=1,j\not=k,j\not=l}^{M} [A_j]^{\nu_{j}} \right)
\end{equation}

\begin{equation}
	\vec{\Psi}_i = \partial_{\vec{C}} \Gamma_i
\end{equation}

%\section{Subtracting Equation}
%
%Let $\vec{V}_1\not=\vec{0}$ and $\vec{V}_2\not=\vec{0}$.
%The smallest weighted difference is:
%\begin{equation}
%	\delta \vec{V} = \dfrac{1}{\left(\vec{V}_1+\vec{V}_2\right)^2} \left[ 
%	\langle \vec{V}_2 | \vec{V}_1+\vec{V}_2 \rangle \vec{V}_1 - \langle \vec{V}_1 | \vec{V}_1+\vec{V}_2 \rangle \vec{V}_2
%	\right]
%\end{equation}

\section{1D-Solving}

\subsection{Solution}

Let us take $\Gamma_i$ (with a least a product or a reactant).
Then there exist a unique $\Xi_i$ for which:
\begin{equation}
\left\lbrace
\begin{array}{rcl}
	\Gamma_i(\vec{C} + \Xi_i \vec{\nu}_i ) & = & 0\\
	 \Xi_i \times \Gamma_i(\vec{C}) &\geq  &0\\
\end{array}
\right.
\end{equation}
 
Indeed, $\Xi_i$ is solution of:
\begin{equation}
	 F_i(\vec{C},X) =  K_i \prod_{j=1}^{M} \left([A_j] - X \nu^r_{ij}\right)^{\nu^r_{ij}} 
	- \prod_{j=1}^{M} \left([A_j] + X \nu^p_{ij} \right)^{\nu^p_{ij}} = 0
\end{equation} 

And each $\Xi_i$ respects all the limiting extents.

 \subsection{Regularization}
We define:
\begin{equation}
\vec{\Psi}_i = \partial_{\vec{C}} \Gamma_i.
\end{equation}
If $|\vec{\Psi}_i|=0$, then we solve $\Gamma_i$. If we still have $|\vec{\Psi}_i|=0$, we declare $\Gamma_i$ blocked.

\subsection{Evolution}

We start from a regularized $\vec{C}$ and some possibly blocked equilibria. We 
want to solve:
\begin{equation}
	\vec{0} = 	\vec{\Gamma}(\vec{C}+\bm{\nu}^T \vec{\xi}) \simeq \vec{\Gamma} + \bm{\Psi} \bm{\nu}^T \vec{\xi}
\end{equation}

\begin{equation}
	\bm{\Omega} = \bm{\Psi} \bm{\nu}^T
\end{equation}	

For a regularized equilibrium, $\Omega_{ii}=\langle \vec{\Psi}_i \vert \vec{\nu}_i \rangle < 0$ is
a scaling factor for $\Gamma_i$, and
\begin{equation}
	\left\vert \dfrac{\Gamma_i}{\langle \vec{\Psi}_i \vert \vec{\nu}_i \rangle} \right\vert
\end{equation}
is homogenous to a concentration.\\
We first estimate:
\begin{equation}
	\vec{\xi} =  {\bm{\Omega}}^{-1} \cdot (-\vec{\Gamma} )
\end{equation}

\subsection{Method}

For each equilibrium,
\begin{equation}
	\Gamma_i(\vec{C}+\Xi_i \vec{\nu}_i) = 0,\;\; \vec{C}_i' = \vec{C}+\Xi_i \vec{\nu}_i
\end{equation}
Each $\vec{C}_i'$ is valid, so is any convex combination of $\vec{C}_i'$.
  
  
\subsection{Method again}
We start from a position $\vec{C}$.
From that position, we compute:
\begin{equation}
	\forall i, \;\;\vec{C}_i' = \vec{C}_i + \Xi_i \vec{\nu}_i,\;\;\Gamma_i(\vec{C}_i') = 0
\end{equation}
For each equilibrium, there is an hyperplane around $\vec{C}_i'$ such that:
\begin{equation}
	\forall i, \;\; \langle \vec{\Psi}_i' \vert \vec{X} - \vec{C}_i'\rangle = 0
\end{equation} 
  
 The distance to the hyperplane is:
 \begin{equation}
 	d_i^2(\vec{X}) = \langle \vec{n}_i \vert \vec{X} - \vec{C}_i'\rangle^2,\;\; \vec{n}_i = \dfrac{1}{\vert  \vec{\Psi}_i' \vert} \vec{\Psi}_i'
 \end{equation} 

\begin{equation}
	D^2 = \dfrac{1}{2} \sum_i   \langle \vec{n}_i \vert \vec{C} - \vec{C}_i'\rangle^2
\end{equation}

so that the decreasing step of $D^2(\vec{C} + \bm{\nu}^T \vec{\xi})$ is:
\begin{equation}
	\vec{\beta} = -\partial_{\vec{\xi}} D^2 = \sum_i \langle \vec{n}_i \vert  \vec{C}_i' - \vec{C}\rangle 
	%\langle \vec{n}_i \vert \bm{\nu}^T
	\cdot \bm{\nu} \vec{n}_i
\end{equation}


 

\end{document}