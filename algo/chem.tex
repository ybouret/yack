\documentclass[aps,12pt]{revtex4}
\usepackage[a4paper]{geometry}
\usepackage{graphicx}
\usepackage{amssymb,amsfonts,amsmath,amsthm}
\usepackage{bm}
\usepackage{pslatex}
\usepackage{mathptmx}
\usepackage{bookman}
\usepackage{chemarr}

 	 
\begin{document}

\section{Notations}

Let $A_1,\ldots,A_M$ be $M$ species involved in $N$ equilibria:
\begin{equation}
	\forall i\in[1:N], \; \sum_j \nu_{i,j} A_j = 0
\end{equation}
and
\begin{equation}
	\forall i\in[1:N], \; 
	\Gamma_i = K_i \prod_{j=1}^{M} [A_j]^{\nu^r_{ij}} 
	- \prod_{j=1}^{M} [A_j]^{\nu^p_{ij}} = 0
\end{equation}

\begin{itemize}
\item The unit of $\Gamma_i$ is $C_0^{\sum_j \nu^p_{ij} }$
\item The unit of $K_i$ is $C_0^{\Delta_r \nu}$
\end{itemize}


\section{Derivatives}
We express the derivative of a product w.r.t. one term:
\begin{equation}
	\partial_{A_k} \left( \prod_{j=1}^{M} [A_j]^{\nu_{j}} \right)  =
	\nu_k [A_k]^{\nu_k-1} \left( \prod_{j=1,j\not=k}^{M} [A_j]^{\nu_{j}} \right)
\end{equation}

Then:
\begin{equation}
	\partial^2_{A_k, A_k} \left( \prod_{j=1}^{M} [A_j]^{\nu_{j}} \right)  =
	\nu_k (\nu_k-1) [A_k]^{\nu_k-2} \left( \prod_{j=1,j\not=k}^{M} [A_j]^{\nu_{j}} \right)
\end{equation}
and:
\begin{equation}
	\partial^2_{A_k\not=A_l} \left( \prod_{j=1}^{M} [A_j]^{\nu_{j}} \right)  =
	\nu_k [A_k]^{\nu_k-1} \nu_l [A_l]^{\nu_l-1}\left( \prod_{j=1,j\not=k,j\not=l}^{M} [A_j]^{\nu_{j}} \right)
\end{equation}

\section{Solving}

\subsection{Equilibrium $[0,1]$}
\begin{equation}
	\xrightleftharpoons[]{} A, \;\; \Gamma = K - [A]
\end{equation}
Starting from $[A]$, the extent $x$ is:
\begin{equation}
	x = K-[A] = \Gamma
\end{equation}

\subsection{Equilibrium $[1,0]$}
\begin{equation}
	A \xrightleftharpoons[]{} , \;\; \Gamma = K[A]-1
\end{equation}
Starting from $[A]$, the extent $x$ is:
\begin{equation}
	x = \dfrac{K[A]-1}{K} = \dfrac{\Gamma}{K}
\end{equation}

\subsection{Equilibrium $[0,n]$}
\begin{equation}
	\xrightleftharpoons[]{} n A, \;\; \Gamma = K - [A]^n
\end{equation}
Starting from $[A]$, the extent $x$ is solution of:
\begin{equation}
	0 = K-([A]+nx)^n \Leftrightarrow x = \dfrac{K^{1/n} - [A]}{n}
\end{equation}



\subsection{Equilibrium $[1,1]$}

\begin{equation}
	A \xrightleftharpoons[]{} B, \;\; \Gamma = K[A] - [B]
\end{equation}
The solving extent is:
\begin{equation}
	x = \dfrac{K[A]-[B]}{1+K} = \dfrac{\Gamma}{1+K}
\end{equation}

\subsection{Equilibrium $[0,[1,1]]$}
\begin{equation}
	\xrightleftharpoons[]{} A+B, \;\; \Gamma = K - [A][B]
\end{equation}
The solving extent is solution of:
\begin{equation}
	0 = K - ([A]+x)([B]+x) = (K - [A][B]) - ([A]+[B]) x - x^2
\end{equation}
so that:
\begin{equation}
	x = \dfrac{\sqrt{4K+([A]-[B])^2} - ([A]+[B])}{2} = \Gamma \dfrac{2}{\sqrt{4K+([A]-[B])^2} + ([A]+[B])}
\end{equation}

\subsection{Equilibrium $[1,[1,1]]$}
\begin{equation}
	A \xrightleftharpoons[]{} B+C, \;\; \Gamma = K[A] - [B][C]
\end{equation}
The solving extent is solution of:
\begin{equation}
	K([A]-x) - ([B]+x)([C]+x)= 0
\end{equation}
The discriminant is:
\begin{equation}
	\Delta = K^2 + ([B]-[C])^2 + 2K(2[A]+[B]+[C])
\end{equation}
The solution is:
\begin{equation}
	x = \dfrac{\sqrt{\Delta}-(K+b+c)}{2} = \Gamma \dfrac{2}{\sqrt{\Delta}+(K+b+c)}
\end{equation}

\subsection{Equilibrium $[1,2]$}
\begin{equation}
	A \xrightleftharpoons[]{} 2B, \;\; \Gamma = K[A] - [B]^2
\end{equation}
The solving extent is solution of:
\begin{equation}
 K([A]-x) - ([B]+2x)^2 = 0
\end{equation}
With:
\begin{equation}
\Delta = 8K[B]+16K[A]+K^2
\end{equation}
we get:
\begin{equation}
	x = \Gamma \dfrac{2}{\sqrt{\Delta} + K+4[B]}
\end{equation}


\subsection{Equilibrium $[ [1,1], [1,1]]$}
\begin{equation}
	A+B \xrightleftharpoons[]{} C+D, \;\; \Gamma = K[A][B] - [C][D]
\end{equation}
The solving extent is solution of:
\begin{equation}
	K([A]-x)([B]-x) - ([C]+x)([D]+x)= 0
\end{equation}
The discriminant is:
\begin{equation}
	\Delta = \left[K([A]+[B])+([C]+[D])\right]^2 + 4 (1-K) \Gamma
\end{equation}

\begin{equation}
x = \Gamma \dfrac{2}{\sqrt{\Delta}+\left[K([A]+[B])+([C]+[D])\right]}
\end{equation}

\section{Evolution}
Let us assume we have an out of equilibria system.
For a given set of concentrations, $\Gamma_i$ is proportional
to the rate of the reaction corresponding to the equilibrium.
For a reduced time $\tau$,
\begin{equation}
	\partial_\tau \vec{C} = \bm{\nu}^T    \bm{\Lambda} \vec{\Gamma}(\vec{C})
\end{equation}
Let us integrate implicitly with $\tau$, since we have the form:
\begin{equation}
	\partial_\tau \vec{C} = \vec{F}(\vec{C})
\end{equation}

\begin{equation}
	\vec{C}_{n+1} = \vec{C}_n +  \tau\left[ \bm{I}_M - \tau \partial_{\vec{C}} \vec{F} \right]^{-1} \vec{F}(\vec{C}_n)
\end{equation}
then
\begin{equation}
	\vec{C}_{n+1} = \vec{C}_n + \tau\left[ \bm{I}_M - \tau  \cdot \bm{\nu}^T   \bm{\Lambda} \bm{\Psi} \right]^{-1}  
	\bm{\nu}^T     \bm{\Lambda} \vec{\Gamma}
\end{equation}


\begin{equation}
	(\bm{\nu}^T \bm{\Psi})_{ii} = \sum_{k=1}^N (\bm{\nu}^T)_{ik} \bm{\Psi}_{kj} =  \sum_{k=1}^N \bm{\nu}_{ki} \bm{\Psi}_{kj}
= \sum_{k=1}^N \bm{\nu}_{ki} \partial_{A_k}\Gamma_i \leq 0
\end{equation}

\begin{equation}
	\forall j \in [1:M], \;\; (\bm{\nu}^T     \bm{\Lambda} \vec{\Gamma})_j = \sum_{k=1}^N (\bm{\nu}^T)_{jk} \lambda_k \Gamma_k = 
	\sum_{k=1}^N (\bm{\nu}^T_{kj}\Gamma_k)  \lambda_k = \sum_{k=1}^N \bm{\Omega}_{jk} \lambda_k
\end{equation}

\begin{equation}
	\forall (j,k) \in [1:M]^2, (\bm{\nu}^T   \bm{\Lambda} \bm{\Psi})_{jk} 
	= \sum_{l=1}^N (\bm{\nu}^T)_{jl} \lambda_l \bm{\Psi}_{lk}
	%= \sum_{l=1}^N (\bm{\nu}_{lj}  \bm{\Psi}_{lk}) \lambda_l
\end{equation}



 

\end{document}