\documentclass[aps,12pt]{revtex4}
\usepackage[a4paper]{geometry}
\usepackage{graphicx}
\usepackage{amssymb,amsfonts,amsmath,amsthm}
\usepackage{bm}
\usepackage{pslatex}
\usepackage{mathptmx}
\usepackage{bookman}
\usepackage{chemarr}

 	 
\begin{document}

\section{Notations}

Let $A_1,\ldots,A_M$ be $M$ species involved in $N$ equilibria:
\begin{equation}
	\forall i\in[1:N], \; \sum_j \nu_{i,j} A_j = 0
\end{equation}
and
\begin{equation}
	\forall i\in[1:N], \; 
	\Gamma_i = K_i \prod_{j=1}^{M} [A_j]^{\nu^r_{ij}} 
	- \prod_{j=1}^{M} [A_j]^{\nu^p_{ij}} = 0
\end{equation}

\begin{itemize}
\item $\Gamma_i$ has the sign of $\Delta_r G_i$.
\item The unit of $\Gamma_i$ is $C_0^{\sum_j \nu^p_{ij} }$
\item The unit of $K_i$ is $C_0^{\Delta_r \nu}$
\end{itemize}


\section{Derivatives}
We express the derivative of a product w.r.t. one term:
\begin{equation}
	\partial_{A_k} \left( \prod_{j=1}^{M} [A_j]^{\nu_{j}} \right)  =
	\nu_k [A_k]^{\nu_k-1} \left( \prod_{j=1,j\not=k}^{M} [A_j]^{\nu_{j}} \right)
\end{equation}

Then:
\begin{equation}
	\partial^2_{A_k, A_k} \left( \prod_{j=1}^{M} [A_j]^{\nu_{j}} \right)  =
	\nu_k (\nu_k-1) [A_k]^{\nu_k-2} \left( \prod_{j=1,j\not=k}^{M} [A_j]^{\nu_{j}} \right)
\end{equation}
and:
\begin{equation}
	\partial^2_{A_k\not=A_l} \left( \prod_{j=1}^{M} [A_j]^{\nu_{j}} \right)  =
	\nu_k [A_k]^{\nu_k-1} \nu_l [A_l]^{\nu_l-1}\left( \prod_{j=1,j\not=k,j\not=l}^{M} [A_j]^{\nu_{j}} \right)
\end{equation}

\begin{equation}
	\vec{\Psi}_i = \partial_{\vec{C}} \Gamma_i
\end{equation}

%\section{Subtracting Equation}
%
%Let $\vec{V}_1\not=\vec{0}$ and $\vec{V}_2\not=\vec{0}$.
%The smallest weighted difference is:
%\begin{equation}
%	\delta \vec{V} = \dfrac{1}{\left(\vec{V}_1+\vec{V}_2\right)^2} \left[ 
%	\langle \vec{V}_2 | \vec{V}_1+\vec{V}_2 \rangle \vec{V}_1 - \langle \vec{V}_1 | \vec{V}_1+\vec{V}_2 \rangle \vec{V}_2
%	\right]
%\end{equation}

\section{1D-Solving}

\subsection{Solution}

Let us take $\Gamma_i$ (with a least a product or a reactant).
Then there exist a unique $\Xi_i$ for which:
\begin{equation}
\left\lbrace
\begin{array}{rcl}
	\Gamma_i(\vec{C} + \Xi_i \vec{\nu}_i ) & = & 0\\
	 \Xi_i \times \Gamma_i(\vec{C}) &\geq  &0\\
\end{array}
\right.
\end{equation}
 
Indeed, $\Xi_i$ is solution of:
\begin{equation}
	 F_i(X) =  K_i \prod_{j=1}^{M} \left([A_j] - X \nu^r_{ij}\right)^{\nu^r_{ij}} 
	- \prod_{j=1}^{M} \left([A_j] + X \nu^p_{ij} \right)^{\nu^p_{ij}} = 0
\end{equation} 


%\begin{equation}
%\begin{array}{rcr}
%	- F'_i(X) & = & 
%	\displaystyle
%	K_i \sum_j 
%	\left[
%	\left(\nu_{ij}^r\right)^2 \left([A_j] - X \nu^r_{ij}\right)^{\nu^r_{ij}-1} 
%	\prod_{k\not=j} \left([A_k] - X \nu^r_{ik}\right)^{\nu^r_{ik}} 
% 	\right]\\
%	\\
%	& + &
%	\displaystyle
%	\sum_j \left[
%	\left(\nu_{ij}^p\right)^2 \left([A_j] + X \nu^p_{ij}\right)^{\nu^p_{ij}-1} 
%	\prod_{k\not=j} \left([A_k] + X \nu^p_{ik}\right)^{\nu^p_{ik}} 
% 	\right]\\
%\end{array}
%\end{equation}


\subsection{Change of each $\Xi$}
If we start from
\begin{equation}
\left\lbrace
\begin{array}{rcl}
	\vec{C}'_i & = & \vec{C} + \Xi_i \vec{\nu}_i\\
	\Gamma_i(\vec{C}'_i) & = & 0\\
\end{array}
\right.
\end{equation}

We look for $\delta \Xi_i$ which keeps $\Gamma_i$ to $0$ for a small perturbation $\delta\vec{C}$:
\begin{equation}
	\Gamma_i\left( \vec{C} + \delta\vec{C} + \Xi_i \vec{\nu}_i + \delta \Xi_i \vec{\nu}_i \right)
\end{equation}
so that we need:
\begin{equation}
	\langle \vec{\Psi}_i' \vert  \delta \Xi_i \vec{\nu}_i + \delta\vec{C} \rangle = 0
\end{equation}
or:
\begin{equation}
	\langle \vec{\Psi}_i' \vert  \vec{\nu}_i  \rangle  \delta \Xi_i = - \langle \vec{\Psi}_i' \vert \delta\vec{C} \rangle 
 \end{equation}
So for an extent $\delta\vec{\xi}$:
\begin{equation}
	\langle \vec{\Psi}_i' \vert  \vec{\nu}_i  \rangle  \delta \Xi_i = - \langle \vec{\Psi}_i' \vert \bm{\nu}^T \vert \delta\vec{\xi} \rangle 
\end{equation}

\subsection{Change of $\vec{\Xi}$}

\begin{equation}
	\vec{\Xi}(\vec{C}+ \bm{\nu}^T  \delta\vec{\xi} ) \simeq  \vec{\Xi}(\vec{C}) -
	\bm{\Omega}
	\delta\vec{\xi}
\end{equation}  
with
\begin{equation}
	\Omega_{ii} = 1
\end{equation}
and
\begin{equation}
\Omega_{i\not=j} = 
\left\lbrace
	\begin{array}{ccll}
	0 & \text{if} & |\langle \vec{\Psi}_i'  \vert \vec{\nu}_i \rangle| = 0 & \text{ (and $\Gamma_i$ is blocked) }\\
	\dfrac{\langle \vec{\Psi}_i'  \vert \vec{\nu}_j \rangle}{\langle \vec{\Psi}_i'  \vert \vec{\nu}_i \rangle}
	 & \text{if} & |\langle \vec{\Psi}_i'  \vert \vec{\nu}_i \rangle| > 0
	 & \\
	\end{array}
\right.
\end{equation}


 	 	
\subsection{Regularization}
We define:
\begin{equation}
\vec{\Psi}_i = \partial_{\vec{C}} \Gamma_i.
\end{equation}
If $|\vec{\Psi}_i|=0$, then we solve $\Gamma_i$. If we still have $|\vec{\Psi}_i|=0$, we declare $\Gamma_i$ blocked.

\subsection{Evolution}

\begin{equation}
	\vec{\Gamma}(\vec{C}+\bm{\nu}^T \vec{\xi}) \simeq \vec{\Gamma} + \bm{\Psi} \bm{\nu}^T \vec{\xi}
\end{equation}	

All the terms of $	\bm{\Psi} \bm{\nu}^T $ are $\leq 0$.




 


 

\end{document}