\documentclass[aps,12pt]{revtex4}
\usepackage[a4paper]{geometry}
\usepackage{graphicx}
\usepackage{amssymb,amsfonts,amsmath,amsthm}
\usepackage{bm}
\usepackage{pslatex}
\usepackage{mathptmx}
\usepackage{bookman}
\usepackage{chemarr}

 	 
\begin{document}

\section{Notations}

Let $A_1,\ldots,A_M$ be $M$ species involved in $N$ equilibria:
\begin{equation}
	\forall i\in[1:N], \; \sum_j \nu_{i,j} A_j = 0
\end{equation}
and
\begin{equation}
	\forall i\in[1:N], \; 
	\Gamma_i = K_i \prod_{j=1}^{M} [A_j]^{\nu^r_{ij}} 
	- \prod_{j=1}^{M} [A_j]^{\nu^p_{ij}} = 0
\end{equation}

\begin{itemize}
\item $\Gamma_i$ has the sign of $\Delta_r G_i$.
\item The unit of $\Gamma_i$ is $C_0^{\sum_j \nu^p_{ij} }$
\item The unit of $K_i$ is $C_0^{\Delta_r \nu}$
\end{itemize}

\section{1D-Solving}
Let us take $\Gamma_i$ (with a least a product or a reactant).
Then there exist a unique $\tilde{\xi}_i$ for which:
\begin{equation}
\left\lbrace
\begin{array}{rcl}
	\Gamma_i(\vec{C} + \vec{\nu}_i \tilde{\xi}_i) & = & 0\\
	 \tilde{\xi}_i \times \Gamma_i(\vec{C}) &\geq  &0\\
\end{array}
\right.
\end{equation}

\section{Derivatives}
We express the derivative of a product w.r.t. one term:
\begin{equation}
	\partial_{A_k} \left( \prod_{j=1}^{M} [A_j]^{\nu_{j}} \right)  =
	\nu_k [A_k]^{\nu_k-1} \left( \prod_{j=1,j\not=k}^{M} [A_j]^{\nu_{j}} \right)
\end{equation}

Then:
\begin{equation}
	\partial^2_{A_k, A_k} \left( \prod_{j=1}^{M} [A_j]^{\nu_{j}} \right)  =
	\nu_k (\nu_k-1) [A_k]^{\nu_k-2} \left( \prod_{j=1,j\not=k}^{M} [A_j]^{\nu_{j}} \right)
\end{equation}
and:
\begin{equation}
	\partial^2_{A_k\not=A_l} \left( \prod_{j=1}^{M} [A_j]^{\nu_{j}} \right)  =
	\nu_k [A_k]^{\nu_k-1} \nu_l [A_l]^{\nu_l-1}\left( \prod_{j=1,j\not=k,j\not=l}^{M} [A_j]^{\nu_{j}} \right)
\end{equation}

\section{Subtracting Equation}

Let $\vec{V}_1\not=\vec{0}$ and $\vec{V}_2\not=\vec{0}$.
The smallest weighted difference is:
\begin{equation}
	\delta \vec{V} = \dfrac{1}{\left(\vec{V}_1+\vec{V}_2\right)^2} \left[ 
	\langle \vec{V}_2 | \vec{V}_1+\vec{V}_2 \rangle \vec{V}_1 - \langle \vec{V}_1 | \vec{V}_1+\vec{V}_2 \rangle \vec{V}_2
	\right]
\end{equation}


\section{Solving}
 
 	
\subsection{Regularization}
We define:
\begin{equation}
\vec{\Psi}_i = \partial_{\vec{C}} \Gamma_i.
\end{equation}
If $|\vec{\Psi}_i|=0$, then we solve $\Gamma_i$. If we still have $|\vec{\Psi}_i|=0$, we declare $\Gamma_i$ blocked.

\subsection{Evolution}

\begin{equation}
	\vec{\Gamma}(\vec{C}+\bm{\nu}^T \vec{\xi}) \simeq \vec{\Gamma} + \bm{\Psi} \bm{\nu}^T \vec{\xi}
\end{equation}	

All the terms of $	\bm{\Psi} \bm{\nu}^T $ are $\leq 0$.




 


 

\end{document}