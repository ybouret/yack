\documentclass[aps,12pt]{revtex4}
\usepackage[a4paper]{geometry}
\usepackage{graphicx}
\usepackage{amssymb,amsfonts,amsmath,amsthm}
\usepackage{bm}
\usepackage{pslatex}
\usepackage{mathptmx}

 	 
\begin{document}

We want to find in $\mathbb{R}^N$:
\begin{equation}
	\vec{F}(\vec{x}) = \vec{0}
\end{equation}
starting from $\vec{x_n}$:
\begin{equation}
	\vec{F}(\vec{x_n}+\vec{h}) \simeq \vec{F} + \bm{J} \cdot \vec{h}
\end{equation}
We define the control function:
\begin{equation}
	f(\vec{x}) = \dfrac{1}{2} \vec{F} \cdot \vec{F}
\end{equation}
and:
\begin{equation}
	f(\vec{x}+\vec{h}) \simeq \dfrac{1}{2}\vec{F}^2 + <\vec{F} | \bm{J} | \vec{h} > + \dfrac{1}{2} <\vec{h}|\bm{J}^t | \bm{J} | \vec{h}>
\end{equation}

The gradient of the control function is:
\begin{equation}
	\vec{\nabla} f = \vec{G} = \bm{J}^t \cdot \vec{F}
\end{equation}

We perform the SVD of $\bm{J}$:
\begin{equation}
	\bm{J} = \bm{U} \bm{W} \bm{V}^t
\end{equation}
and we define the pseudo-inverse of $\bm{J}$ for a given fractional tolerance:
\begin{equation}
	\tilde{\bm{J}} = \bm{V} \tilde{\bm{W}} \bm{U}^t
\end{equation}	
We compute the step:
\begin{equation}
	\vec{S} = - \tilde{\bm{J}} \cdot \vec{F} 
\end{equation}
and we define the initial slope $\sigma$:
\begin{equation}
	\sigma = -\vec{S}\vec{G} = <\vec{F}| \bm{U} \bm{W}\tilde{\bm{W}}\bm{U}|\vec{F}> \geq 0
\end{equation}
If $\sigma\leq 0$, then the point is singular.

\begin{equation}
f(x) = f_a - \sigma x + \beta x^2 + \gamma x^3
\end{equation}

With $b<c$:
\begin{equation}
\begin{bmatrix}
	b^2 & b^3 \\
	c^2 & c^3 \\
\end{bmatrix}
\begin{bmatrix}
	\beta\\
	\gamma\\
\end{bmatrix}
=
\begin{bmatrix}
	f_b - f_a + \sigma b\\
	f_c - f_a + \sigma c\\
\end{bmatrix}
\end{equation}

\begin{equation}
	\begin{bmatrix}
	\beta\\
	\gamma\\
\end{bmatrix}
= \dfrac{1}{b^2c^3-b^3c^2} 
\begin{bmatrix}
	c^3 & -b^3 \\
	-c^2 & b^2 \\
\end{bmatrix}
\begin{bmatrix}
	f_b - f_a + \sigma b\\
	f_c - f_a + \sigma c\\
\end{bmatrix}
\end{equation}


\begin{equation}
	\begin{bmatrix}
	\beta\\
	\gamma\\
\end{bmatrix}
= \dfrac{1}{b^2c^2(c-b)} 
\begin{bmatrix}
	c^3 \delta_b -  b^3 \delta_c + \sigma (bc^3-cb^3)\\
	b^2 \delta_c - c^2 \delta_b + \sigma (cb^2 - b c ^2)\\
\end{bmatrix}
\end{equation}

We use $b=\alpha c,\;\alpha<1$,

\begin{equation}
	\Lambda = b^2c^2(c-b) > 0
\end{equation}

	\begin{equation}
	\begin{bmatrix}
	\beta\\
	\gamma\\
\end{bmatrix}
= \dfrac{1}{c^3(1-\alpha)\alpha^2} 
\begin{bmatrix}
	c^3 \delta_b -  c^3 \alpha^3 \delta_c + c^3 \sigma (\alpha-\alpha^3)\\
	c^2 \alpha^3 \delta_c - c^2 \delta_b + \sigma c^2 (\alpha^2 - \alpha)\\
\end{bmatrix}
\end{equation}


\begin{equation}
	f'(x) = -\sigma + 2 \beta x + 3 \gamma x^2
\end{equation}

\begin{equation}
	\Delta' = \beta^2 + 3\gamma\sigma = \dfrac{1}{\Lambda^2} \left( \beta'^2 + 3 \gamma' \sigma \Lambda \right)
\end{equation}

\begin{equation}
	x_\pm = \dfrac{-\beta \pm \sqrt{\beta^2 + 3\gamma\sigma}}{3\gamma}
\end{equation}


\end{document}