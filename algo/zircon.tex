\documentclass[aps,12pt]{revtex4}
\usepackage[a4paper]{geometry}
\usepackage{graphicx}
\usepackage{amssymb,amsfonts,amsmath,amsthm}
\usepackage{bm}
\usepackage{pslatex}
\usepackage{mathptmx}

 	 
\begin{document}

We want to find in $\mathbb{R}^N$:
\begin{equation}
	\vec{F}(\vec{x}) = \vec{0}
\end{equation}
starting from $\vec{x_n}$:
\begin{equation}
	\vec{F}(\vec{x_n}+\vec{h}) \simeq \vec{F} + \bm{J} \cdot \vec{h}
\end{equation}
We define the control function:
\begin{equation}
	f(\vec{x}) = \dfrac{1}{2} \vec{F} \cdot \vec{F}
\end{equation}
and:
\begin{equation}
	f(\vec{x}+\vec{h}) \simeq \dfrac{1}{2}\vec{F}^2 + <\vec{F} | \bm{J} | \vec{h} > + \dfrac{1}{2} <\vec{h}|\bm{J}^t | \bm{J} | \vec{h}>
\end{equation}

The gradient of the control function is:
\begin{equation}
	\vec{\nabla} f = \vec{G} = \bm{J}^t \cdot \vec{F}
\end{equation}

We perform the SVD of $\bm{J}$:
\begin{equation}
	\bm{J} = \bm{U} \bm{W} \bm{V}^t
\end{equation}
and we define the pseudo-inverse of $\bm{J}$ for a given fractional tolerance:
\begin{equation}
	\tilde{\bm{J}} = \bm{V} \tilde{\bm{W}} \bm{U}^t
\end{equation}	
We compute the step:
\begin{equation}
	\vec{S} = - \tilde{\bm{J}} \cdot \vec{F} 
\end{equation}
and we define the initial slope $\sigma$:
\begin{equation}
	\sigma = -\vec{S}\vec{G} = <\vec{F}| \bm{U} \bm{W}\tilde{\bm{W}}\bm{U}|\vec{F}> \geq 0
\end{equation}
If $\sigma\leq 0$, then the point is singular.

\begin{equation}
f(x) = f_a - \sigma x - \beta x^2 +  \gamma  x^3
\end{equation}

 
\begin{equation}
\begin{bmatrix}
	-\dfrac{c^2}{4} & \dfrac{c^3}{8} \\
	-c^2 & c^3 \\
\end{bmatrix}
\begin{bmatrix}
	\beta\\
	\gamma\\
\end{bmatrix}
=
\begin{bmatrix}
	f_b - f_a + \sigma \dfrac{c}{2}\\
	f_c - f_a + \sigma c\\
\end{bmatrix}
\end{equation}

\begin{equation}
	\begin{bmatrix}
	\beta\\
	\gamma\\
\end{bmatrix}
= \dfrac{8}{c^3} 
\begin{bmatrix}
	-c & \frac{c}{8} \\
	-1 & \frac{1}{4} \\
\end{bmatrix}
\begin{bmatrix}
	\delta_b   + \sigma \frac{c}{2}\\
	\delta_c   + \sigma c\\
\end{bmatrix}
= \dfrac{1}{c^3} 
\begin{bmatrix}
	-8c & c \\
	-8  & 2 \\
\end{bmatrix}
\begin{bmatrix}
	\delta_b   + \sigma \frac{c}{2}\\
	\delta_c   + \sigma c\\
\end{bmatrix}
\end{equation}


\begin{equation}
	\begin{bmatrix}
	\beta\\
	\gamma\\
\end{bmatrix}
= \dfrac{1}{c^3} 
\begin{bmatrix}
	 c\left[\delta_c - 8\delta_b - 3 \sigma c \right]\\
	 2\left[\delta_c - 4\delta_b - \sigma c \right]\\
\end{bmatrix}
\end{equation}

The minimum is at:
\begin{equation}
	x_+ = \dfrac{\beta+\sqrt{\beta^2+3\sigma\gamma}}{3\gamma}
\end{equation}
Using $\beta=\dfrac{\beta'}{c^2},\gamma=\dfrac{\gamma'}{c^3}$:
 	 
\begin{equation}
	x_+ = \dfrac{\beta'+\sqrt{\beta'^2+3 (\sigma c) \gamma'}}{3\gamma'} c
\end{equation}

\end{document}