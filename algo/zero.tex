\documentclass[aps,10pt]{revtex4}
\usepackage[a4paper]{geometry}
\usepackage{graphicx}
\usepackage{amssymb,amsfonts,amsmath,amsthm}
\usepackage{bm}
\usepackage{pslatex}
\usepackage{mathptmx}
\usepackage{bookman}
 	 
\begin{document}
We have a continuous function $f(x)$ on $[a;c]$ such that:
$$f(a)f(c)<0.$$
Let us define:
$$
	b = \dfrac{a+c}{2}
$$
We define $h(x)=f(x)e^{\alpha(x-b)}$ that linearizes $f$ on the interval.
$$
\left\lbrace
\begin{array}{rcl}
	h(a) & = & f(a)e^{\alpha(a-b)}\\
	h(b) & = & f(b)\\
	h(c) & = & f(c) e^{\alpha(c-b)}\\
	h(b) & = & \dfrac{h(a)+h(c)}{2}\\
\end{array}
\right.
$$
We define
$$
Q = e^{\alpha\frac{c-a}{2}}
$$
to get
$$	
\left\lbrace
\begin{array}{rcl}
	h(a) & = & f(a)/Q\\
 	h(c) & = & f(c)Q\\
\end{array}
\right.
$$
The factor $Q$ is defined by:
$$
	f(a)/Q + f(c)Q - 2f(b) = 0
$$
or:
$$
	Q^2 f(c) - 2Qf(b) + f(a) = 0 
$$
We get:
$$
	\Delta' = f^2(b) - f(a)f(c), \;\; \sqrt{\Delta'} > |f(b)|
$$
so that:
$$
	Q = \dfrac{f(b) + \mathrm{sign}[ f(c) ] \sqrt{\Delta'}}{f(c)} = \dfrac{f(b) + \epsilon_c \sqrt{\Delta'}}{f(c)}
$$
and:
$$
	\dfrac{1}{Q} = \frac{f(b) - \epsilon_c \sqrt{\Delta'}}{f(a)}
$$
so that:
$$
\begin{array}{rcccl}
	h(a) & = & f(a)/Q & = & f(b) - \epsilon_c \sqrt{\Delta'}\\
	h(c) & = & Qf(c)  & = & f(b) + \epsilon_c \sqrt{\Delta'}\\
\end{array}
$$
We compute the zero position estimate:
$$
	z = a - (c-a) \dfrac{h(a)}{h(c)-h(a)} = a + (c-a) \dfrac{\epsilon_c \sqrt{\Delta'} - f(b)}{2 \epsilon_c \sqrt{\Delta'}}
$$
then
$$
	z = b - \dfrac{(c-a)}{2} \dfrac{\epsilon_c f(b)}{\sqrt{\Delta'}} = b + \left(\dfrac{-\mathrm{sign}[ f(c) ]}{2}\right) (c-a) \dfrac{f(b)}{\sqrt{\Delta'}}
$$
 
\end{document}
